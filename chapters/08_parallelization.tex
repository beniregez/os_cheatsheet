\subsubsection*{8) OpenMP \& Parallelization}
Parallelization maps computation on data to parallel/distributed systems respecting dependencies. Key steps: analyze program parallelism, identify \textbf{task/data parallelism}, decompose \textbf{granularity} (fine, medium, coarse), and consider communication cost.

Dependencies: \textbf{control} (task order), \textbf{data} (flow, anti, output), \textbf{name} dependencies. Mapping: assign subtasks spatially (processor) and temporally (when), \textbf{static/dynamic} scheduling.

OpenMP: \textbf{shared-memory API} for multi-threading via \#pragma omp, runtime libs, env vars (OMP\_NUM\_THREADS). \textbf{Simplifies} parallel code, portable across multicore, NUMA, GPUs. Parallel regions execute copies of code; loops via \#pragma omp parallel for. Threads share vars, needing \textbf{synchronization} (critical, atomic, barriers) to avoid \textbf{races}. Functions: \texttt{omp\_set\_num\_threads()}, \texttt{omp\_get\_thread\_num()}, \texttt{omp\_in\_parallel()}, \texttt{omp\_get\_num\_procs()}, \texttt{omp\_set\_dynamic()}. Use \textbf{private}, \textbf{shared}, \textbf{reduction} clauses.

Limitations: OpenMP is \textbf{shared-memory only}, lower parallel efficiency vs MPI for distributed memory.
Performance: needs enough parallelism, balanced load, data locality, minimize synchronization.
Overhead: communication, cache effects, false sharing, load imbalance.
Tools: \textbf{OMP\_NUM\_THREADS}, \textbf{OMP\_SCHEDULE}, \textbf{parallel loops}, critical sections, \textbf{barriers}.

\textbf{Speedup limit:} Amdahl’s Law $S = 1 / (f + (1 - f) / N)$ (f=serial fraction). \textbf{Scalability:} impacted by sequential part, load imbalance, communication.
