\subsubsection*{6a) File System Interface}
FS = abstraction: stores data+progs, hides phys. storage. File = ADT by OS: seq of records (bytes/lines/etc.), e.g., ELF. User view: smallest unit on disk. Attributes: name, id, type, loc (ptr), size, timestamps, perms. Pos ptr per process for R/W; all ops (exc. create/del) $\to$ open.\\
Ops: create (alloc+dir), open (name+perms), read/write (handle), seek, delete (free), truncate.\\
Access: seq (read\_next, write\_next), direct (read(n), write(n)), indexed (index in RAM).\\
Dir: single-level (global list), 2-level (per user), tree (abs/rel path), acyclic-graph (links$\to$files, resolve). Dir+files = on disk. No cycles: file-only links, GC, or detect on link add. Dir ops: search, create, delete, list, rename, traverse.\\
Mem-mapped = alt to syscalls (open/R/W); shared mem via mmap.

\subsubsection*{6b) FS Implementation}
FS enable access to storage (disks/NVM). Disks: blocks w/ sectors (512/4096 B).
FS design balances UI \& algorithms mapping logical FS to physical. Layered arch.: app → logical FS (metadata, dirs, FCBs) → file org (files, logical blocks, free space) → basic FS → I/O ctrl → device drivers/interrupts. Pseudo-FS (e.g., /proc in Linux, at /proc) = virtual FS in kernel mem, file interface for system info (e.g., /proc/cpuinfo, /proc/[pid]/status, /proc/meminfo), contents generated on access. Pseudo-FS $\neq$ Gen-FS: no backing store, dynamic data, mostly r/o. Gen-FS: persistent, modifiable.  
FS types: UNIX FS, NTFS, FAT, ext3/4.  
On-disk/in-mem data: boot block (bootloader = blocks loading kernel from FS), volume ctrl block (superblock/MFT: \#blocks, block size, free-block count/pointers), dirs, FCBs. Mount = attach volume (e.g., /dev/sda1) at mount point (dir); volumes may span partitions (e.g., LVM). Mounting checks valid FS via device driver reading device dir.  
Caches: dir info, open file tables (global + per-process w/ FCBs), block buffers.  
Name lookup: linear list (slow), hash table (faster, collisions), ordered list (linked), B+ trees.  
Block alloc: contiguous (fast, needs size, external frag, compaction), linked (pointers, no external frag, sequential access), indexed (index block(s), random access, overhead, handles via linked/multilevel/combined).  
Free-space mgmt: bitmap (bit=free/used), free list (linked blocks, no contiguous alloc), grouping (list of free blocks per block).

\subsubsection*{6c) FS Internals}
Storage devices split into partitions (e.g., /dev/sda1), each mountable as a volume (FS-formatted). Raw partition = no FS; cooked = has FS. Root partition: OS, bootloader specifies (loads kernel from FS). Other partitions: other OS, FS, user data; mounted auto/manual (e.g., /dev/sda1 at /home).  
Mount = associate volume w/ mount point (dir in tree). Volumes must be mounted before access; OS verifies FS via device driver.